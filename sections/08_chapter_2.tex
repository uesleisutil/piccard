% !TEX root = /media/ueslei/Ueslei/INPE/PCI/Projetos/Guia_COAWST/main.tex
\chapterimage{header.jpg} % Chapter heading image

\chapter{COAWST in a parallel cluster}\index{COAWST na Kerana}
\bigskip

\section{About our cluster Kerana}
\bigskip

 The CRAY XE machine, also called Kerana, is a cluster with parallel architecture with 84 processing nodes and 2688 cores and is located at CPTEC/INPE,
in Cachoeira Paulista, São Paulo. For having the ability to parallelize operations through the MPI interface, the cluster is used to run our numerical models.
\bigskip

\section{Signing up a user account}
\bigskip

 To start the process of applying for a user account in Kerana, it is necessary to use the network at INPE with a fixed IP in your computer. Your advisor or supervisor 
is required to send an email to \textit{Helpdesk} (\textcolor{bleu_cite}{\href{mailto:helpdesk.cptec@inpe.br}{\textit{helpdesk.cptec@inpe.br}}}) informing the MAC address, hostname
and reason for the request.
\bigskip

 Now you have in hands the fixed IP, contact the \textit{Helpdesk} (\textcolor{bleu_cite}{\href{mailto:helpdesk.cptec@inpe.br}{\textit{helpdesk.cptec@inpe.br}}}) to request a form 
to use Kerana.
\bigskip

\section{Signing in to the Kerana cluster}\label{keranaacess}
\bigskip

 Access will be done entirely through your operating system terminal. You will use two primary commands: \textit{ssh} for accessing and manipulating files 
and folders and \textit{sftp} to download and upload files.
\bigskip

 To access and modify files and folders, type in the terminal, replacing \textit{name.surname} with the user provided by \textit{Helpdesk}:
\bigskip

\begin{tcolorbox}[enhanced,
  grow to left by   = 0cm,
  grow to right by  = 0cm,
  enlarge top by    = 0cm,
  enlarge bottom by = 0cm,
  tcbox raise base,
  boxrule           = 1.0pt,
  left              = 18mm,
  colframe          = red!50!black,coltext=red!25!black,colback=red!10!white,
  overlay           = {\begin{tcbclipinterior}\fill[red!75!blue!50!white] (frame.south west)
    rectangle node[text=white,font=\sffamily\bfseries\footnotesize,rotate=0] {WARNING} ([xshift=18mm]frame.north west);\end{tcbclipinterior}}]
    From now on, whenever the guide shows the user as \textit{name.surname}, change to your username provided by the Helpdesk.
\end{tcolorbox}
\bigskip

\begin{bashcode}
ssh -Y name.surname@acesso-hpc.cptec.inpe.br -p 2000
\end{bashcode}
\bigskip

 To download and/or upload, use:
\bigskip

\begin{bashcode}
sftp -P2000 name.surname@acesso-hpc.cptec.inpe.br
\end{bashcode}
\bigskip

\begin{tcolorbox}[enhanced,
  grow to left by   = 0cm,
  grow to right by  = 0cm,
  enlarge top by    = 0cm,
  enlarge bottom by = 0cm,
  tcbox raise base,
  boxrule           = 1.0pt,
  left              = 18mm,
  colframe          = red!50!black,coltext=red!25!black,colback=red!10!white,
  overlay           = {\begin{tcbclipinterior}\fill[red!75!blue!50!white] (frame.south west)
    rectangle node[text=white,font=\sffamily\bfseries\footnotesize,rotate=0] {WARNING} ([xshift=18mm]frame.north west);\end{tcbclipinterior}}]
It is not possible to download and upload multiple files at the same time using \textit{sftp}. One advice is to compress into a single \textit{.tar} or \textit{.zip} file and then uncompress them.
\end{tcolorbox}
\bigskip

 To add files from your computer to Kerana:
\bigskip

\begin{bashcode}
put file.tar.gz
\end{bashcode}
\bigskip

 To extract Kerana files to your computer:
\bigskip

\begin{bashcode}
get file.tar.gz
\end{bashcode}
\bigskip

\section{File repository}\label{reposit}
\bigskip

 Certain files are needed in each user's area to make more easy to work with the cluster. You will find them in the directory:
\bigskip

\begin{bashcode}
/scratch/adriano.sutil/repositorio/
\end{bashcode}
\bigskip

 To copy the files to your area, type:
\bigskip

\begin{bashcode}
cp -r /scratch/adriano.sutil/repositorio /scratch/name.surname
\end{bashcode}
\bigskip

\begin{tcolorbox}[enhanced,
  grow to left by   = 0cm,
  grow to right by  = 0cm,
  enlarge top by    = 0cm,
  enlarge bottom by = 0cm,
  tcbox raise base,
  boxrule           = 1.0pt,
  left              = 18mm,
  colframe          = red!50!black,coltext=red!25!black,colback=red!10!white,
  overlay           = {\begin{tcbclipinterior}\fill[red!75!blue!50!white] (frame.south west)
    rectangle node[text=white,font=\sffamily\bfseries\footnotesize,rotate=0] {WARNING} ([xshift=18mm]frame.north west);\end{tcbclipinterior}}]
    From now on this guide will use the files that are inside this repository, so it is essential that they are in your area.
\end{tcolorbox}
\bigskip

\section{Kerana environment}
\bigskip

 It is necessary to activate some modules in the cluster to compile COAWST. In this case, open the file \textit{.bashrc} which is located at the root directory of your user. 
\bigskip

\begin{bashcode}
vim .bashrc
\end{bashcode}
\bigskip

 Add the following commands at the end of the file, changing only \textit{name.surname} to your username:
\bigskip


\begin{bashcode}
module load java
module load netcdf

export PATH=/scratch/name.surname/repositorio/Softs/nedit/5.5:$PATH
export PATH=/scratch/name.surname/repositorio/Softs/bin:$PATH

export PHDF5=${HDF_DIR}
export WRFIO_NCD_LARGE_FILE_SUPPORT=1

export PATH=/home/luciano.pezzi/local/bin:$PATH
export JASPERINC=/home/luciano.pezzi/local/include
export JASPERLIB=/home/luciano.pezzi/local/lib
export LD_LIBRARY_PATH=/home/luciano.pezzi/local/lib:$LD_LIBRARY_PATH
\end{bashcode}
\bigskip

 Save and type in the terminal:
\bigskip

\begin{bashcode}
source .bashrc
\end{bashcode}

\section{Downloading COAWST}\label{coawstbaixa}
\bigskip

\begin{tcolorbox}[enhanced,
  grow to left by=0cm,%   equivalent to negative mdframed 'leftmargin'
  grow to right by=0cm,%  equivalent to negative mdframed 'rightmargin'
  enlarge top by=0cm,%     equivalent to mdframed 'skipabove'
  enlarge bottom by=0cm,%  equivalent to mdframed 'skipbelow'
  tcbox raise base,
  boxrule=1.0pt,
  left=18mm,
  colframe=yellow!50!black,coltext=red!25!black,colback=yellow!10!white,
  overlay={\begin{tcbclipinterior}\fill[yellow!75!blue!50!white] (frame.south west)
    rectangle node[text=white,font=\sffamily\bfseries\footnotesize,rotate=0] {ATTENTION} ([xshift=18mm]frame.north west);\end{tcbclipinterior}}]
COAWST v3.6 is already in the repository within the Kerana cluster, as discussed in Section \textcolor{bleu_cite}{\ref{reposit}}.
\end{tcolorbox}
\bigskip

 To download COAWST, send an email to Dr. John Warner (\textcolor{bleu_cite}{\href{mailto:jcwarner@usgs.gov}{\textit{jcwarner@usgs.gov}}}), one of the heads behind the coupled regional modeling system.
\bigskip

 After access is granted, with the user credentials and password provided by Dr. John Warner, type in the command the command below, changing the \ textit {myusrname} to your username.
\bigskip

\begin{bashcode}[fontsize=\footnotesize]
svn checkout --username myusrname https://coawstmodel.sourcerepo.com/coawstmodel/COAWST
\end{bashcode}
\bigskip

 Add the COAWST folder to your Kerana desktop using \textit{sftp}, as in the Section \textcolor{bleu_cite}{\ref{keranaacess}}.
\bigskip

\section{Automating the compilation of COAWST in Kerana}\label{autowrf}
\bigskip

\begin{tcolorbox}[enhanced,
  grow to left by=0cm,%   equivalent to negative mdframed 'leftmargin'
  grow to right by=0cm,%  equivalent to negative mdframed 'rightmargin'
  enlarge top by=0cm,%     equivalent to mdframed 'skipabove'
  enlarge bottom by=0cm,%  equivalent to mdframed 'skipbelow'
  tcbox raise base,
  boxrule=1.0pt,
  left=18mm,
  colframe=red!50!black,coltext=red!25!black,colback=red!10!white,
  overlay={\begin{tcbclipinterior}\fill[red!75!blue!50!white] (frame.south west)
    rectangle node[text=white,font=\sffamily\bfseries\footnotesize,rotate=0] {WARNING} ([xshift=18mm]frame.north west);\end{tcbclipinterior}}]
  This guide uses COAWST v3.6.
\end{tcolorbox}
\bigskip

 To speed up the process, it is possible to automate some compilation steps. Enter the directory:
\bigskip

\begin{bashcode}
cd /scratch/name.surname/COAWST/WRF/arch
\end{bashcode}
\bigskip

 Open file \textit{Config.pl}:
\bigskip

\begin{bashcode}
nedit Config.pl
\end{bashcode}
\bigskip

 Search for the lines:
\bigskip

\begin{bashcode}[fontsize=\small]
printf "\nEnter selection [%d-%d] : ",1,$opt ;
$response = <STDIN> ;
\end{bashcode}
\bigskip

 And replace \textit{<STDIN>} to 42, as in the example below:
\bigskip

\begin{bashcode}
printf "\nEnter selection [%d-%d] : ",1,$opt ;
$response = 42 ;
\end{bashcode}
\bigskip

 Open file \textit{configure.defaults}:
\bigskip

\begin{bashcode}
nedit configure.defaults
\end{bashcode}
\bigskip

 Look for the \textit{TRADEFLAG} option, which is on line 1262, approximately:
\bigskip

\begin{bashcode}
TRADEFLAG = CONFIGURE_TRADEFLAG
\end{bashcode}
\bigskip

 And modify to:
\bigskip

\begin{bashcode}
TRADEFLAG = -traditional
\end{bashcode}
\bigskip

 These modifications will select the configurations of the Kerana cluster (\textit{CRAY CCE (ftn/gcc): Cray XE and XC (dmpar)}), among those available to use COAWST, 
 as shown in Figure \textcolor{bleu_cite}{\ref{compskerana}}.
\bigskip

\begin{figure}[H]
  \centering
  \includegraphics[width=0.80\textwidth]{linuxoptions.png}
  \caption{Computational options available for selection when compiling COAWST.}
  \label{compskerana}
\end{figure}
\bigskip

 Now, open the \textit{Config\_new.pl} file and then look for the following lines:
\bigskip

\begin{bashcode}[fontsize=\footnotesize]
printf "Compile for nesting? (1=basic, 2=preset moves, 3=vortex following) [default 1]: " ;
}
$response = <STDIN> ;
\end{bashcode}
\bigskip

 And change \textit{<STDIN>} to the basic WRF atmospheric model nesting mode, as in the following example:
\bigskip

\begin{bashcode}[fontsize=\footnotesize]
printf "Compile for nesting? (1=basic, 2=preset moves, 3=vortex following) [default 1]: " ;
}
$response = 1 ;
\end{bashcode}
\bigskip

\section{Compiling the MCT}
\bigskip

\begin{tcolorbox}[enhanced,
  grow to left by=0cm,%   equivalent to negative mdframed 'leftmargin'
  grow to right by=0cm,%  equivalent to negative mdframed 'rightmargin'
  enlarge top by=0cm,%     equivalent to mdframed 'skipabove'
  enlarge bottom by=0cm,%  equivalent to mdframed 'skipbelow'
  tcbox raise base,
  boxrule=1.0pt,
  left=18mm,
  colframe=red!50!black,coltext=red!25!black,colback=red!10!white,
  overlay={\begin{tcbclipinterior}\fill[red!75!blue!50!white] (frame.south west)
    rectangle node[text=white,font=\sffamily\bfseries\footnotesize,rotate=0] {WARNING} ([xshift=18mm]frame.north west);\end{tcbclipinterior}}]
    This guide uses COAWST v 3.6.
\end{tcolorbox}
\bigskip


 Each new user must compile the MCT before compiling COAWST. The first step is to use the \textit{setup\_pgi.sh} file. This file was copied from the previous 
repository folder. So open the file:
\bigskip

\begin{bashcode}
nedit setup_pgi.sh
\end{bashcode}
\bigskip

 If necessary, change the directories according to the name of your COAWST folder and execute the file to load the modules:
\bigskip

\begin{bashcode}
source setup_pgi.sh
\end{bashcode}
\bigskip

 The libraries will be activated, as shown in Figure \textcolor{bleu_cite}{\ref{modulos}}:
\bigskip


\begin{figure}[H]
    \centering
    \includegraphics[width=0.85\textwidth]{modules.png}
    \caption{Modules activated with the file \textit{setup\_pgi.sh}.}
    \label{modulos}
\end{figure}
\bigskip

 Enter the MCT folder directory:
\bigskip

\begin{bashcode}
cd /home/name.surname/COAWST/Lib/MCT
\end{bashcode}
\bigskip

 Open the file \textit{Makefile.conf}:
\bigskip

\begin{bashcode}
nedit Makefile.conf
\end{bashcode}
\bigskip

 And modify the file as the following:
\bigskip

\begin{tcolorbox}[enhanced,
  grow to left by=0cm,%   equivalent to negative mdframed 'leftmargin'
  grow to right by=0cm,%  equivalent to negative mdframed 'rightmargin'
  enlarge top by=0cm,%     equivalent to mdframed 'skipabove'
  enlarge bottom by=0cm,%  equivalent to mdframed 'skipbelow'
  tcbox raise base,
  boxrule=1.0pt,
  left=18mm,
  colframe=yellow!50!black,coltext=yellow!25!black,colback=yellow!10!white,
  overlay={\begin{tcbclipinterior}\fill[yellow!75!blue!50!white] (frame.south west)
    rectangle node[text=white,font=\sffamily\bfseries\footnotesize,rotate=0] {ATTENTION} ([xshift=18mm]frame.north west);\end{tcbclipinterior}}]
    Remember to change the \textit{name.surname}!
\end{tcolorbox}
\bigskip

\begin{bashcode}
FC  	    = ftn
FCFLAGS	 = -O2
F90FLAGS        = 
REAL8           = -r8
ENDIAN          = -Mbyteswapio
INCFLAG         = -I
INCPATH         =
MPILIBS         = 
DEFS            = -DSYSLINUX -DCPRPGI
FPP	     = cpp
FPPFLAGS        = -P -C -N -traditional
CC              = cc
ALLCFLAGS       = -DFORTRAN_UNDERSCORE_ -DSYSLINUX -DCPRPGI -O
COMPILER_ROOT   = 
BABELROOT       = 
PYTHON          = 
PYTHONOPTS      = 
FORT_SIZE       = 
CRULE           = .c.o
90RULE          = .F90.o
F90RULECPP      = .F90RULECPP
INSTALL         = /home/name.surname/COAWST/Lib/MCT/install-sh -c
MKINSTALLDIRS   = /home/name.surname/COAWST/Lib/MCT/mkinstalldirs
abs_top_builddir= /home/name.surname/COAWST/Lib/MCT/
MCTPATH         = /home/name.surname/COAWST/Lib/MCT/mct
MPEUPATH        = /home/name.surname/COAWST/Lib/MCT/mpeu
EXAMPLEPATH     = /home/name.surname/COAWST/Lib/MCT/examples
MPISERPATH      = 
libdir          = /home/name.surname/COAWST/Lib/MCT/pgi/lib
includedir      = /home/name.surname/COAWST/Lib/MCT/pgi/include
AR	      = ar cq
RM	      = rm -f
\end{bashcode}
\bigskip

 Install the MCT by typing the following commands:
\bigskip

\begin{bashcode}
make
make install
\end{bashcode}
\bigskip
 Observe the messages that appear in the terminal and look for errors. If not, the MCT was successfully compiled.
\bigskip


\section{Compiling the Sandy test case}
\bigskip

 There are some test cases within COAWST to be compiled and worked on. In this case we will compile Hurricane Sandy’s project,
which couples and nests WRF, ROMS and SWAN. First, it is necessary to know the structure of COAWST files and folders.
\bigskip

 The typical COAWST directory structure is exemplified in Figure \textcolor{bleu_cite}{\ref{pastascoa}}. Mainly, we will use the folders \textit{Projects} e \textit{Work} to work on.
\bigskip

\begin{figure}[H]
    \centering
    \includegraphics[width=0.50\textwidth]{estruturacoawst.png}
    \caption{
Representation of the main COAWST folder and subfolders.}
    \label{pastascoa}
\end{figure}
\bigskip

\subsection{Projects folder}\index{Pasta Projetcs}
\bigskip

 In order to organize the projects, the \textit{COAWST/Projects/Sandy} folder contain all the files used to simulate the
Sandy case. The following files must be inside the folder:
\bigskip

\begin{itemize}
\item Bound\_spec\_command
\item coastline.mat
\item coupling\_sandy.in
\item create\_sandy\_application.m
\item hycom\_info.mat
\item ijcoast.mat
\item multi\_1.at\_10m.dp.201210.grb2
\item multi\_1.at\_10m.hs.201210.grb2
\item multi\_1.at\_10m.tp.201210.grb2
\item namelist.input
\item ocean\_sandy.in
\item roms\_master\_climatology\_sandy.m
\item roms\_narr\_Oct2012.nc
\item roms\_narr\_ref3\_Oct2012.nc
\item Rweigths.txt
\item Sandy\_bdy.nc
\item Sandy\_clm.nc
\item Sandy\_clm\_ref3.nc
\item sandy.h
\item Sandy\_ini.nc
\item Sandy\_ini\_ref3.nc
\item Sandy\_init.hot
\item Sandy\_ref3\_init.hot
\item Sandy\_roms\_contact.nc
\item Sandy\_roms\_grid.nc
\item Sandy\_roms\_grid\_ref3.nc
\item Sandy\_swan\_bathy.bot
\item Sandy\_swan\_bathy\_ref3.bot
\item Sandy\_swan\_coord.grd
\item Sandy\_swan\_coord\_ref3.grd
\item scrip\_sandy\_moving.nc
\item scrip\_sandy\_static.nc
\item specpts.mat
\item swan\_narr\_Oct2012.dat
\item swan\_narr\_ref3\_Oct2012.dat
\item swan\_sandy.in
\item swan\_sandy\_ref3.in
\item tide\_forc\_Sandy.nc
\item TPAR10.txt
\item TPAR11.txt
\item TPAR12.txt
\item TPAR13.txt
\item TPAR14.txt
\item TPAR15.txt
\item TPAR16.txt
\item TPAR17.txt
\item TPAR18.txt
\item TPAR1.txt
\item TPAR2.txt
\item TPAR3.txt
\item TPAR4.txt
\item TPAR5.txt
\item TPAR6.txt
\item TPAR7.txt
\item TPAR8.txt
\item TPAR9.txt
\item USeast\_bathy.mat
\item Uweigths.txt
\item Vweigths.txt
\item wrfbdy\_d01
\item wrfinput\_d01
\item wrfinput\_d02
\item wrflowinp\_d01
\item wrflowinp\_d02
\end{itemize}
\bigskip

\subsection{Work folder}\index{Pasta Work}\label{workcoawstsec}
\bigskip

 To make the management of projects easier, it is suggested, for each new user, the creation of the \textit{Work} folder in
main COAWST directory, with each project inserted separately within it. It is in this folder that test case will be simulated.
\bigskip

\begin{bashcode}
cd /scratch/name.surname/COAWST
mkdir Work
cd Work
mkdir Sandy
\end{bashcode}
\bigskip

 The folder \textit{/scratch/name.surname/COAWST/Work/Sandy} must contain the following files.
\bigskip

\begin{itemize}
\item run\_sandy.sh
\item limpa.sh
\item link.sh
\end{itemize}
\bigskip

 The file \textit{run\_sandy.sh} will be used to start the simulation, the \textit{link.sh} creates symbolic links in the \textit{Work} folder, 
that will be used by the models, and \textit{limpa.sh} is used to clear the folder if an error occurs and a new integration needs to be started. The files 
are inside the folder \textit{/repositorio/work\_coawst}.
\bigskip

 Therefore:

\bigskip

\begin{bashcode}
cd /scratch/name.surname/repositorio/work_coawst
cp limpa.sh link.sh run_sandy.sh /scratch/name.surname/COAWST/Work/Sandy
\end{bashcode}
\bigskip

 Go to the \textit{/scratch/name.surname/COAWST/Work/Sandy} folder and open the file \textit{run\_sandy.sh}:
\bigskip

\begin{bashcode}
cd /scratch/name.surname/COAWST/Work/Sandy
nedit run_sandy.sh
\end{bashcode}
\bigskip

 Search for the following commands and modify the directories according to your username:
\bigskip

\begin{bashcode}
PBS -o /scratch/name.surname/COAWST/Work/Sandy/rws_total.out
ROOTDIR=/scratch/name.surname/COAWST
\end{bashcode}
\bigskip


\subsection{Compiling the test case}\index{Compilar o caso teste Sandy}
\bigskip

 Go to the Sandy project folde:
\bigskip

\begin{bashcode}
/home/name.surname/COAWST/Projects/Sandy
\end{bashcode}
\bigskip

 Open the following files to make changes for the next steps:
\bigskip

\begin{itemize}
\item coupling\_sandy.in
\item swan\_sandy.in
\item swan\_sandy\_ref3.in
\item ocean\_sandy.in
\end{itemize}
\bigskip

\begin{bashcode}
nedit coupling_sandy.in swan_sandy.in swan_sandy_ref3.in ocean_sandy.in
\end{bashcode}
\bigskip

 Look for the command line below in the file \textit{coupling\_sandy.in}:
\bigskip

\begin{bashcode}
OCN_name = Projects/Sandy/ocean_sandy.in
\end{bashcode}
\bigskip

 And replace with:
\bigskip

\begin{bashcode}
OCN_name = ocean_sandy.in
\end{bashcode}
\bigskip

 In the files \textit{swan\_sandy.in} and \textit{swan\_sandy\_ref3.in} complete all directory paths below:
\bigskip

 Modify:
\bigskip

\begin{bashcode}
READGRID COORDINATES 1 'Projects/Sandy/Sandy_swan_coord.grd' 4 0 0 FREE
READINP BOTTOM 1 'Projects/Sandy/Sandy_swan_bathy.bot' 4 0 FREE
&READINP WIND 1 'Projects/Sandy/swan_namnarr_30Sep10Nov2012.dat' 4 0 FREE
INITIAL HOTSTART SINGLE 'Projects/Sandy/Sandy_init.hot'
\end{bashcode}
\bigskip

 By:
\bigskip

\begin{bashcode}[fontsize=\scriptsize]
READGRID COORDINATES 1 '/scratch/name.surname/COAWST/Projects/Sandy/Sandy_swan_coord.grd' 4 0 0 FREE
READINP BOTTOM 1 '/scratch/name.surname/COAWST/Projects/Sandy/Sandy_swan_bathy.bot' 4 0 FREE
&READINP WIND 1 '/scratch/name.surname/COAWST/Projects/Sandy/swan_namnarr_30Sep10Nov2012.dat' 4 0 FREE
INITIAL HOTSTART SINGLE '/scratch/name.surname/COAWST/Projects/Sandy/Sandy_init.hot'
\end{bashcode}
\bigskip

 In the file \textit{ocean\_sandy.in} look for the command lines like below:
\bigskip

\begin{bashcode}
MyAppCPP = SANDY
VARNAME  = ROMS/External/varinfo.dat
GRDNAME == Projects/Sandy/Sandy_roms_grid.nc \
           Projects/Sandy/Sandy_roms_grid_ref3.nc
ININAME == Projects/Sandy/Sandy_ini.nc \
           Projects/Sandy/Sandy_ini_ref3.nc
NGCNAME =  Projects/Sandy/Sandy_roms_contact.nc
BRYNAME == Projects/Sandy/Sandy_bdy.nc
FRCNAME == Projects/Sandy/roms_narr_Oct2012.nc \
           Projects/Sandy/roms_narr_ref3_Oct2012.nc
\end{bashcode}
\bigskip

 And replace with:
\bigskip

\begin{bashcode}[fontsize=\footnotesize]
MyAppCPP = Sandy
VARNAME  = /scratch/name.surname/COAWST/ROMS/External/varinfo.dat
GRDNAME == /scratch/name.surname/COAWST/Projects/Sandy/Sandy_roms_grid.nc \
           /scratch/name.surname/COAWST/Projects/Sandy/Sandy_roms_grid_ref3.nc
ININAME == /scratch/name.surname/COAWST/Projects/Sandy/Sandy_ini.nc \
           /scratch/name.surname/COAWST/Projects/Sandy/Sandy_ini_ref3.nc
NGCNAME =  /scratch/name.surname/COAWST/Projects/Sandy/Sandy_roms_contact.nc
BRYNAME == /scratch/name.surname/COAWST/Projects/Sandy/Sandy_bdy.nc
FRCNAME == /scratch/name.surname/COAWST/Projects/Sandy/roms_narr_Oct2012.nc \
           /scratch/name.surname/COAWST/Projects/Sandy/roms_narr_ref3_Oct2012.nc
\end{bashcode}
\bigskip

 Go back to the main COAWST folder and open the file \textit{coawst.bash}:
\bigskip

\begin{bashcode}
cd /scratch/name.surname/COAWST
nedit coawst.bash
\end{bashcode}
\bigskip

 Search for the following commands and modify, if necessary:
\bigskip

\begin{bashcode}
export   COAWST_APPLICATION=JOE_TC
export   MY_ROOT_DIR=${HOME}/COAWST
export   MY_HEADER_DIR=${MY_PROJECT_DIR}/Projects/JOE_TC
\end{bashcode}
\bigskip

 By:
\bigskip

\begin{bashcode}
export   COAWST_APPLICATION=Sandy
export   MY_ROOT_DIR=${HOME}/COAWST
export   MY_HEADER_DIR=${MY_PROJECT_DIR}/Projects/Sandy
\end{bashcode}
\bigskip

 Optionally, activate again the modules into the file \textit{setup\_pgi.sh}, and then compile the project with the command:
\bigskip

\begin{bashcode}
./coawst.bash -j 4  1> coawst.pgi.sandy 2>&1 &
\end{bashcode}
\bigskip

 This command will create the text file \textit{coawst.pgi.sandy} where you can follow the compilation progress. Open the text file with the following 
command and search for the final message, as shown in Figure \textcolor{bleu_cite}{\ref{compcoafinal}}.
\bigskip

\begin{bashcode}
nedit coawst.pgi.sandy
\end{bashcode}
\bigskip

\begin{figure}[H]
    \centering
    \includegraphics[width=0.65\textwidth]{coawstfinal.png}
    \caption{Final message after compiling COAWST.}
    \label{compcoafinal}
\end{figure}
\bigskip

 If you desire to follow the compilation progress through the terminal, use the command:
\bigskip

\begin{bashcode}
tail -f coawst.pgi.sandy
\end{bashcode}
\bigskip

\begin{tcolorbox}[enhanced,
  grow to left by=0cm,%   equivalent to negative mdframed 'leftmargin'
  grow to right by=0cm,%  equivalent to negative mdframed 'rightmargin'
  enlarge top by=0cm,%     equivalent to mdframed 'skipabove'
  enlarge bottom by=0cm,%  equivalent to mdframed 'skipbelow'
  tcbox raise base,
  boxrule=1.0pt,
  left=18mm,
  colframe=yellow!50!black,coltext=yellow!25!black,colback=yellow!10!white,
  overlay={\begin{tcbclipinterior}\fill[yellow!75!blue!50!white] (frame.south west)
    rectangle node[text=white,font=\sffamily\bfseries\footnotesize,rotate=0] {ATTENTION} ([xshift=18mm]frame.north west);\end{tcbclipinterior}}]
The compilation process can take several minutes, even hours. 
\end{tcolorbox}
\bigskip


 We finished! In the main COAWST directory, a file called \textit{coawstM} will be created. In this file is stored all the information about your project. 
 Now with COAWST compiled, we will start the test case.
\bigskip

\section{Simulating the Sandy Test Case}\label{sandyexec}
\bigskip

 To simulate the test case, search for the \textit{run\_sandy.sh} file in \textit{Work/Sandy} folder. Type:
\bigskip

\begin{bashcode}
cd /scratch/name.surname/COAWST/Work/Sandy
nedit run_sandy.sh
\end{bashcode}
\bigskip

\begin{tcolorbox}[enhanced,
  grow to left by=0cm,%   equivalent to negative mdframed 'leftmargin'
  grow to right by=0cm,%  equivalent to negative mdframed 'rightmargin'
  enlarge top by=0cm,%     equivalent to mdframed 'skipabove'
  enlarge bottom by=0cm,%  equivalent to mdframed 'skipbelow'
  tcbox raise base,
  boxrule=1.0pt,
  left=18mm,
  colframe=red!50!black,coltext=red!25!black,colback=red!10!white,
  overlay={\begin{tcbclipinterior}\fill[red!75!blue!50!white] (frame.south west)
    rectangle node[text=white,font=\sffamily\bfseries\footnotesize,rotate=0] {WARNING} ([xshift=18mm]frame.north west);\end{tcbclipinterior}}]
    The \textit{Work} folder should contain the files \textit{clean.sh}, \textit{link.sh} and \textit{run\_sandy.sh}. They can be found in the folder used as a repository. See Section \textcolor{bleu_cite} {\ref{workcoawstsec}}.
\end{tcolorbox}
\bigskip

 When opening the file, check if the directories are in accordance with your username and type the command below to start the integration.
\bigskip

\begin{bashcode}
qsub run_sandy.sh
\end{bashcode}
\bigskip

\begin{tcolorbox}[enhanced,
  grow to left by=0cm,%   equivalent to negative mdframed 'leftmargin'
  grow to right by=0cm,%  equivalent to negative mdframed 'rightmargin'
  enlarge top by=0cm,%     equivalent to mdframed 'skipabove'
  enlarge bottom by=0cm,%  equivalent to mdframed 'skipbelow'
  tcbox raise base,
  boxrule=1.0pt,
  left=18mm,
  colframe=green!50!black,coltext=green!25!black,colback=green!10!white,
  overlay={\begin{tcbclipinterior}\fill[green!75!blue!50!white] (frame.south west)
    rectangle node[text=white,font=\sffamily\bfseries\footnotesize,rotate=0] {NOTE} ([xshift=18mm]frame.north west);\end{tcbclipinterior}}]
 The command \textit{qsub} will submit you job. It will send your project to a batch of the cluster, reserving a part of the processors for your simulation.
\end{tcolorbox}
\bigskip

 The process will generate two files to follow the evolution of the simulation: \textit{log.out} and \textit{log.err}. To open it, use:
\bigskip

\begin{bashcode}
nedit log.out log.err
\end{bashcode}
\bigskip

 Or look directly at the terminal with the command:
\bigskip

\begin{bashcode}
tail -f log.out
\end{bashcode}
\bigskip

 The outputs of the simulations will be stored in the \textit{Work/Sandy} folder. If an error occurs, clean the Work folder with the command:
\bigskip

\begin{bashcode}
./limpa.sh
\end{bashcode}
\bigskip


\begin{tcolorbox}[enhanced,
  grow to left by=0cm,%   equivalent to negative mdframed 'leftmargin'
  grow to right by=0cm,%  equivalent to negative mdframed 'rightmargin'
  enlarge top by=0cm,%     equivalent to mdframed 'skipabove'
  enlarge bottom by=0cm,%  equivalent to mdframed 'skipbelow'
  tcbox raise base,
  boxrule=1.0pt,
  left=18mm,
  colframe=yellow!50!black,coltext=yellow!25!black,colback=yellow!10!white,
  overlay={\begin{tcbclipinterior}\fill[yellow!75!blue!50!white] (frame.south west)
    rectangle node[text=white,font=\sffamily\bfseries\footnotesize,rotate=0] {ATTENTION} ([xshift=18mm]frame.north west);\end{tcbclipinterior}}]
    The simulation the may take several hours.
\end{tcolorbox}
\bigskip
