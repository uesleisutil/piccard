% !TEX root = /media/ueslei/Ueslei/INPE/PCI/Guia_COAWST/main.tex
\newpage
~\vfill
\thispagestyle{empty}

\begin{figure}[H]
    \centering
    \vspace*{\fill}
    \includegraphics[width=0.7\textwidth]{loa.jpg}
    \vspace{2.7cm}
\end{figure}


\noindent \textsc{Ueslei Adriano Sutil}

\noindent \textcolor{bleu_cite}{\textit{uesleisutil1@gmail.com}}\\ % URL

\noindent \textsc{Luciano Ponzi Pezzi}

\noindent \textcolor{bleu_cite}{\textit{luciano.pezzi@inpe.br}}\\ % URL

\noindent Produzido através do Programa de Capacitação Interna do MCTIC/CNPq (Processo 301110/2017-4). \\

\noindent Conteúdo licenciado sob \textit{Creative Commons Attribution-NonCommercial 3.0 Unported} (CC BY-NC 3.0). Para obter uma cópia da licença, acesse: \textit{\href{http://creativecommons.org/licenses/by-nc/3.0}}{\color{bleu_cite}{\textit{http://creativecommons.org/licenses/by-nc/3.0}}}. A figura A Grande Onda de Kanagawa, que compõe o cabeçalho de cada capítulo, e de autoria de Katsushika Hokusai, está disponível para domínio público.\\

\noindent Template Legrand Orange Book produzido por Mathias Legrand (\textcolor{bleu_cite}{\textit{legrand.mathias@gmail.com}}) e modificado por Vel (\textcolor{bleu_cite}{\textit{vel@latextemplates.com}}) e Andrea Hidalgo. Licenciado sob \textit{Creative Commons Attribution-NonCommercial 3.0 Unported} (CC BY-NC 3.0).\\

\noindent \textit{Versão 2.0}  % Printing/edition date

\noindent \textit{Julho 2019} \\ % Printing/edition date

\begin{figure}[H]
    \centering
    \vspace*{\fill}
    \includegraphics[width=0.27\textwidth]{cnpq.png}
    \hspace{9.2cm}
    \includegraphics[width=0.15\textwidth]{inpe.jpg}
\end{figure}
