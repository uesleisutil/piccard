% !TEX root = /media/ueslei/Ueslei/INPE/PCI/Projetos/Guia_COAWST/main.tex
\chapterimage{header.jpg}
\chapter{COAWST features}
\bigskip
 So far, we learned how to use the Kerana cluster and how to compile and run a COAWST test case.
From now on we will enter into the specifics of the models, such as changing the number of processors
and the information exchange rate between them.
\bigskip

\section{Structural model files}
\bigskip

 As you may have noticed when simulating the Sandy test case, the models have files that assist the user about how to setup the project.

\bigskip

 ROMS uses the \textit{sandy.h} file as a file containing the C pre-processing options that define the project. 
 It also uses the \textit{ocean\_sandy.in} as standard input for
running the model. This file defines the spatial dimensions of the project and parameters that are not informed during compilation,
such as time steps, coefficients and physical constants, configuration of vertical coordinates, flags for
control the output frequency, among other factors. You can learn more about this file by accessing the
website\textcolor{bleu_cite}{\textit{}\footnote{\textcolor{bleu_cite}{\href{https://www.myroms.org/wiki/ocean.in}{https://www.myroms.org/wiki/ocean.in}}}}.
\bigskip

 WRF uses the file \textit{namelist.input} to manage information about the project, as well as
physical parameterization schemes that will be used. To learn about the file description,
visit the NOAA portal website\textcolor{bleu_cite}{\textit{}\footnote{\textcolor{bleu_cite}{\href{https://esrl.noaa.gov/gsd/wrfportal/namelist\_input\_options.html}{https://esrl.noaa.gov/gsd/wrfportal/namelist\_input\_options.html}}}}.
To learn about the physical options of the model and the references of each one, visit the
WRF website\textcolor{bleu_cite}{\textit{}\footnote{\textcolor{bleu_cite}{\href{http://www2.mmm.ucar.edu/wrf/users/phys\_references.html}{http://www2.mmm.ucar.edu/wrf/users/phys\_references.html}}}}.
\bigskip

 SWAN uses the file \textit{swan\_sandy.in} as a manager. It describes several parameters,
such as the project description, the input data, the grid and the initial and boundary conditions, the
wave physics parameterizations, among others. To learn more about configuring the file, visit the \textit{User Manual} tab
in the SWAN website\textcolor{bleu_cite}{\textit{}\footnote{\textcolor{bleu_cite}{\href{http://swanmodel.sourceforge.net/}{http://swanmodel.sourceforge.net/}}}}.
and search for the section \textit{Description of commands}.
\bigskip

\section{Modifying the number of processors}
\bigskip

\subsection{ROMS}
\bigskip

 The processors are located in the \textit{ocean.in} file, which is located in the \textit{Projects} folder. They
are called \textit{NtileI} and \textit{NtileJ}. An example is in Figure \textcolor{bleu_cite}{\ref{romsproc}}.
In this case, ROMS will reserve 320 processors to run, since 16 x 20 = 320.

\bigskip

\begin{figure}[H]
    \centering
    \includegraphics[width=0.80\textwidth]{roms_proc.png}
    \caption{Representation of the number of processors used in ROMS.}
    \label{romsproc}
\end{figure}
\bigskip

\subsection{WRF}
\bigskip

 To change the number of WRF processors, it is necessary to modify the \textit{namelist.input} file.
Search for the variables \textit{nproc\_x} and \textit{nproc\_y}, as in Figure \textcolor{bleu_cite}{\ref{procswrf}}.
In this case, 320 processors will be reserved for the atmospheric model.

\begin{figure}[H]
    \centering
    \includegraphics[width=0.50\textwidth]{wrf_proc.png}
    \caption{Representation of the number of processors for the WRF.}
    \label{procswrf}
\end{figure}
\bigskip

\subsection{SWAN}

\bigskip The SWAN \textit{.in} file does not have the numbers of processors. In that case just change the number
number of processors in \textit{coupling.in}, as shown in the next \textcolor{bleu_cite}{\ref{coawstproc}} subsection.
\bigskip

\subsection{COAWST}\label{coawstproc}
\bigskip

 Now that the number of processors have been modified, the coupler needs to be informed about the change. 
This information is communicated through the file \textit{coupling.in}. It should contain the total number of processors 
to be used by the ROMS, WRF and SWAN models, as in the figure \textcolor{bleu_cite}{\ref{procscoa}}:
\bigskip

\begin{figure}[H]
    \centering
    \includegraphics[width=0.80\textwidth]{taxa_acopl.png}
    \caption{Representation of the number of processors for each COAWST module.}
    \label{procscoa}
\end{figure}
\bigskip

 The \textit{NnodesATM} refers to the total number of processors used by WRF, \textit{NnodesWAV} is the
total used by SWAN and \textit{NnodesOCN} is total used by ROMS. Change according to the total number of processors used
in the previous steps.
\bigskip

 Finally, it is necessary to modify the total number of processors in \textit{run.sh},used to submit the experiment.
Add the total number of processors used by the three models, following the equation \textcolor{bleu_cite}{\ref{equacao2}}:
\bigskip

\begin{equation}
TotalProc = NnodesATM + NnodesWAV + NnodesOCN
\label{equacao2}
\end{equation}
\bigskip

 Where\textit{TotalProc} is the sum of all processors used in the models.
\bigskip

 Now open the file \textit{run.sh}, located inside the folder \textit{Work}, and look for the following line:
\bigskip

\begin{bashcode}
\# PBS -l mppwidth= 3
\end{bashcode}
\bigskip

 And for the line:
\bigskip

\begin{bashcode}
aprun -n 36 coawstM ./coupling.in 1> log.out 2> log.err
\end{bashcode}
\bigskip

 Modify the number \textit{3} by the total number of processors used.
\bigskip

\section{Modifying the coupling time interval between models}
\bigskip

 To modify the information exchanged between models, open the file \textit{coupling.in} and
modify the variables \textit{TI\_ATM2WAV}, \textit{TI\_ATM2OCN}, \textit{TI\_WAV2ATM}, \textit{TI\_WAV2OCN}, \textit{TI\_OCN2WAV},
          \textit{TI\_OCN2ATM}, as in Figure \textcolor{bleu_cite}{\ref{taxaacopla}}.
\bigskip

\begin{tcolorbox}[enhanced,
  grow to left by=0cm,%   equivalent to negative mdframed 'leftmargin'
  grow to right by=0cm,%  equivalent to negative mdframed 'rightmargin'
  enlarge top by=0cm,%     equivalent to mdframed 'skipabove'
  enlarge bottom by=0cm,%  equivalent to mdframed 'skipbelow'
  tcbox raise base,
  boxrule=1.0pt,
  left=18mm,
  colframe=yellow!50!black,coltext=yellow!25!black,colback=yellow!10!white,
  overlay={\begin{tcbclipinterior}\fill[yellow!75!blue!50!white] (frame.south west)
    rectangle node[text=white,font=\sffamily\bfseries\footnotesize,rotate=0] {ATTENTION} ([xshift=18mm]frame.north west);\end{tcbclipinterior}}]
    The information exchange rate is defined in seconds.
\end{tcolorbox}
\bigskip

\begin{figure}[H]
    \centering
    \includegraphics[width=0.95\textwidth]{coupltime.png}
    \caption{Information exchange interval between models used in COAWST. In this example, the exchange will occur every 900 seconds.}
    \label{taxaacopla}
\end{figure}
\bigskip
