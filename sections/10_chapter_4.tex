% !TEX root = /media/ueslei/Ueslei/INPE/PCI/Projetos/Guia_COAWST/main.tex
\chapterimage{header.jpg}
\chapter{Building the WRF}\index{Construindo o WRF}
\bigskip 
 Now that we have learned how to simulate a test case and change the coupling rate and number of processors, we will begin 
the process of creating the grid and forcing files for the WRF using the NCEP Climate Forecast System Reanalysis (CFSR; \cite{Saha2006}).
\bigskip

\section{Compiling WRF in Kerana cluster}
\bigskip
 We recommend to copy the WRF folder that is inside COAWST to your work root area, in order to avoid conflicts
if you choose to use the WRF without coupling to the COAWST model. Therefore:\bigskip

\begin{bashcode}
cd /scratch/name.surname/COAWST
cp -r WRF /scratch/nome.dsobrenome
\end{bashcode}
\bigskip

 Activate the modules in the \textit{setup\_pgi.sh} file with the command:
\bigskip
\begin{bashcode}
source setup_pgi.sh
\end{bashcode}
\bigskip

 Enter the copy folder of the WRF (\textit{/home/name.surname/WRF}) and run:
\bigskip

\begin{bashcode}
./configure
\end{bashcode}
\bigskip

\begin{tcolorbox}[enhanced,
  grow to left by=0cm,%   equivalent to negative mdframed 'leftmargin'
  grow to right by=0cm,%  equivalent to negative mdframed 'rightmargin'
  enlarge top by=0cm,%     equivalent to mdframed 'skipabove'
  enlarge bottom by=0cm,%  equivalent to mdframed 'skipbelow'
  tcbox raise base,
  boxrule=1.0pt,
  left=18mm,
  colframe=green!50!black,coltext=green!25!black,colback=green!10!white,
  overlay={\begin{tcbclipinterior}\fill[green!75!blue!50!white] (frame.south west)
    rectangle node[text=white,font=\sffamily\bfseries\footnotesize,rotate=0] {NOTE} ([xshift=18mm]frame.north west);\end{tcbclipinterior}}]
The next step can be automated. If you choose the automated compilation, see Section \textcolor{bleu_cite}{\ref{autowrf}}.
\end{tcolorbox}
\bigskip

 If you have automated the compilation process, as shown in Section \textcolor{bleu_cite}{\ref{autowrf}}, the files \textit{real.exe}, \textit {wrf.exe},
\textit{tc.exe} and \textit{ndown.exe} will be created. If you choose the manual option on the Kerana cluster, look down for the option \textit{Cray XC CLE / Linux x86 \_64, Cray compiler with gcc (dmpar)}.
\bigskip

 In the next option, the following message will appear:
\bigskip

\begin{bashcode}
Compile for nesting? (1=basic, 2=preset moves, 3=vortex following) [default 1]:
\end{bashcode}
\bigskip

 Choose option 1.
\bigskip

\begin{tcolorbox}[enhanced,
  grow to left by=0cm,%   equivalent to negative mdframe 'leftmargin'
  grow to right by=0cm,%  equivalent to negative mdframed 'rightmargin'
  enlarge top by=0cm,%     equivalent to mdframed 'skipabove'
  enlarge bottom by=0cm,%  equivalent to mdframed 'skipbelow'
  tcbox raise base,
  boxrule=1.0pt,
  left=18mm,
  colframe=red!50!black,coltext=red!25!black,colback=red!10!white,
  overlay={\begin{tcbclipinterior}\fill[red!75!blue!50!white] (frame.south west)
    rectangle node[text=white,font=\sffamily\bfseries\footnotesize,rotate=0] {WARNING} ([xshift=18mm]frame.north west);\end{tcbclipinterior}}]
End of the automated build process.
\end{tcolorbox}
\bigskip

 Type:
\bigskip

\begin{bashcode}
compile em_real
\end{bashcode}
\bigskip

 The \textit{real.exe}, \textit{wrf.exe}, \textit{tc.exe} and \textit{ndown.exe} will be generated in the \textit{/home/name.surname/WRF/test/em\_real}.
\bigskip

\section{WRF Preprocessing System (WPS)}\label{wpssecao}
\bigskip

 To build the initial and boundary conditions of the WRF, we will use WRF Preprocessing System (WPS). The WPS is a set of three programs that 
prepare the input data for the simulations. It consists of the following modules:
\bigskip

\begin{itemize}
\item \textbf{\textit{geogrid}}: Defines the model domain and interpolates the geographic data for the grid;
\item \textbf{\textit{ungrib}}: Extract weather fields from \textit{.grib} files;
\item \textbf{\textit{metgrid}}: Interpolates the weather fields extracted by ungrib to the model grid defined by metgrid.
\end{itemize}
\bigskip

 WPS is controlled by the \textit{namelist.wps} file and is outlined in Figure \textcolor{bleu_cite}{\ref{wpsdetalha}}. You can find more information in the WPS 
website\textcolor{bleu_cite}{\textit{}\footnote{\textcolor{bleu_cite}{\href{http://www2.mmm.ucar.edu/wrf/OnLineTutorial/}{http://www2.mmm.ucar.edu/wrf/OnLineTutorial/}}}}.
\bigskip

 After creating these files, the \textit{real.exe} is used to interpolate the meteorological fields to \texteta{}   level.
\bigskip

\begin{figure}[H]
    \centering
    \includegraphics[width=0.55\textwidth]{image002.png}
    \caption{WPS schemeatics. \newline Author: \textcite{duda2006}.}
    \label{wpsdetalha}
\end{figure}
\bigskip

\subsection{Downloading WPS}
\bigskip

 WPS is included in the COAWST package, however it can be downloaded by visiting the WPS
\textit{website}\textcolor{bleu_cite}{\footnote{\textcolor{bleu_cite}{\href{http://www2.mmm.ucar.edu/wrf/users/download/get\_source.html}{http://www2.mmm.ucar.edu/wrf/users/download/get\_source.html}}}}. 
Download it and add the \textit{tar.gz} file inside the cluster with \textit{sftp}:
\bigskip

\begin{bashcode}
sftp -P2000 name.surname@acesso-hpc.cptec.inpe.br
cd COAWST
put WPSV3.9.0.1.tar.gz
\end{bashcode}
\bigskip

 To uncompress, use \textit{ssh} to enter the cluster and type:
\bigskip

\begin{bashcode}
ssh -Y name.surname@acesso-hpc.cptec.inpe.br -p 2000
cd COAWST
tar -xvzf WPSV3.9.0.1.tar.gz
\end{bashcode}
\bigskip

\subsection{Compiling WPS in the Kerana cluster}\label{wpsker}
\bigskip

\begin{tcolorbox}[enhanced,
    grow to left by=0cm,%   equivalent to negative mdframed 'leftmargin'
    grow to right by=0cm,%  equivalent to negative mdframed 'rightmargin'
    enlarge top by=0cm,%     equivalent to mdframed 'skipabove'
    enlarge bottom by=0cm,%  equivalent to mdframed 'skipbelow'
    tcbox raise base,
    boxrule=1.0pt,
    left=18mm,
    colframe=red!50!black,coltext=red!25!black,colback=red!10!white,
    overlay={\begin{tcbclipinterior}\fill[red!75!blue!50!white] (frame.south west)
      rectangle node[text=white,font=\sffamily\bfseries\footnotesize,rotate=0] {WARNING} ([xshift=18mm]frame.north west);\end{tcbclipinterior}}]
      As stated in Section \textcolor{bleu_cite}{\ref{wpsker}}, WPS is included with COAWST. We recommend to use that version.
\end{tcolorbox}
\bigskip

 To compile the WPS and generate the executables, it is necessary to activate the libraries with \textit{setup\_pgi.sh} file
found in the \textit{/home/name.surname/repository} directory. Activate them with the command:
\bigskip

\begin{bashcode}
source setup_pgi.sh
\end{bashcode}
\bigskip

 Inside the WPS folder, enter the following command:
\bigskip

\begin{bashcode}
./configure
\end{bashcode}
\bigskip

 You will be asked which machine and compiler is being used. For the Kerana users, choose the option 
\textit{Cray XE / XC CLE / Linux x86 \_64, PGI compiler (serial)}. In this case, a message will be displayed that \textit{Fortran is not compatible 
with C and NetCDF}, but ignore it.
\bigskip

 Open \textit{configure.wps} file:
\bigskip

\begin{bashcode}
nedit configure.wps
\end{bashcode}
\bigskip

 Modify:
\bigskip

\begin{bashcode}
WRF_DIR  = ../WRFV3
SFC      = ftn
SCC      = gcc
DM_CC    = cc
DM_FC    = ftn
FFLAGS   = -N255 -f free -h byteswapio
F77FLAGS = -N255 -f fixed -h byteswapio

\end{bashcode}
\bigskip

 By:
\bigskip

\begin{bashcode}
WRF_DIR  = /scratch/name.surname/WRF
SFC      = ftn
SCC      = gcc
DM_CC    = gcc
DM_FC    = ftn
FFLAGS   =
F77FLAGS =
\end{bashcode}
\bigskip

 Save the changes and start the compilation with the command:
\bigskip

\begin{bashcode}
./compile
\end{bashcode}
\bigskip

 You should generate the executables \textit{metgrid.exe} and \textit{geogrid.exe}. However, it is possible that \textit{ungrib.exe} 
is not generated. In this case, repeat:
\bigskip

\begin{bashcode}
./configure
\end{bashcode}
\bigskip

 Choose the option \textit{Linux x86\_64, PGI compiler (dmpar)}.
\bigskip

 Open \textit{configure.wps} file again, modify:
\bigskip

\begin{bashcode}
WRF_DIR = ../WRFV3
SFC     = pgf90
SCC     = pgcc
DMFC    = mpif90
DMCC    = mpicc
\end{bashcode}
\bigskip

 By:
\bigskip

\begin{bashcode}
WRF_DIR = /home/name.surname/WRF
SFC     = ftn
SCC     = gcc
DMFC    = ftn
DMCC    = gcc
\end{bashcode}
\bigskip

 The installer will generate only the \textit{ungrib.exe}.

\section{Building your project using CFSR}
\bigskip

 NCEP Climate Forecast System Reanalysis 
(CFSR)\textcolor{bleu_cite}{\textit{}\footnote{\textcolor{bleu_cite}{\href{http://rda.ucar.edu/datasets/ds093.0}{http://rda.ucar.edu/datasets/ds093.0}}}}
a reanalysis, avaliable from January 1979 to December 2010, with a coupled model (atmosphere-ocean-land surface and sea ice) that
assimilates data from remote sensing, research cruises, oceanographic buoys and weather stations. This database
has vertical resolution with 38 levels, extending from the surface to 1 hPa, with a 6-hour temporal resolution
and horizontal of 0.5\degree for pressure levels data and 0.312\degree for variables on the surface.
\bigskip

 To access the data it is necessary to register on the site. Then, access the database link,
click on \textit{Data Acess} and then, \textit{Get a subset}.
\bigskip

 On the next website page, as seen in Figure \textcolor{bleu_cite}{\ref{detalhacfsr}}, write the temporal selection
and select in \textit {Parameter presets} one of the three WRF presets: \textit{Pressure},
\textit{Surface} and \textit{SST}. 
\bigskip

\begin{tcolorbox}[enhanced,
    grow to left by=0cm,%   equivalent to negative mdframed 'leftmargin'
    grow to right by=0cm,%  equivalent to negative mdframed 'rightmargin'
    enlarge top by=0cm,%     equivalent to mdframed 'skipabove'
    enlarge bottom by=0cm,%  equivalent to mdframed 'skipbelow'
    tcbox raise base,
    boxrule=1.0pt,
    left=18mm,
    colframe=red!50!black,coltext=red!25!black,colback=red!10!white,
    overlay={\begin{tcbclipinterior}\fill[red!75!blue!50!white] (frame.south west)
      rectangle node[text=white,font=\sffamily\bfseries\footnotesize,rotate=0] {WARNING} ([xshift=18mm]frame.north west);\end{tcbclipinterior}}]
      Only one parameter (\textit{Pressure}, \textit{Surface} and \textit{SST}) can be downloaded at a time, therefore you need to repeat this step three times.
\end{tcolorbox}
\bigskip

\begin{figure}[H]
    \centering
    \includegraphics[width=0.95\textwidth]{1.png}
    \caption{Screenshot of the CFSR data page.}
    \label{detalhacfsr}
\end{figure}
\bigskip

 Download the data (in \textit{GRIB} format) and separate it into three folders according to the parameters of each one:
SST, Surface and Pressure. Compress them and put them in the cluster:
\bigskip

\begin{tcolorbox}[enhanced,
    grow to left by=0cm,%   equivalent to negative mdframed 'leftmargin'
    grow to right by=0cm,%  equivalent to negative mdframed 'rightmargin'
    enlarge top by=0cm,%     equivalent to mdframed 'skipabove'
    enlarge bottom by=0cm,%  equivalent to mdframed 'skipbelow'
    tcbox raise base,
    boxrule=1.0pt,
    left=18mm,
    colframe=red!50!black,coltext=red!25!black,colback=red!10!white,
    overlay={\begin{tcbclipinterior}\fill[red!75!blue!50!white] (frame.south west)
      rectangle node[text=white,font=\sffamily\bfseries\footnotesize,rotate=0] {WARNING} ([xshift=18mm]frame.north west);\end{tcbclipinterior}}]
      Remember to download in GRIB format, otherwise WPS will not recognize it.
\end{tcolorbox}
\bigskip

\begin{bashcode}
tar -cvzf SST.tar.gz SST/
tar -cvzf Pressure.tar.gz Pressure/
tar -cvzf Surface.tar.gz Surface/
ssh -Y name.surname@acesso-hpc.cptec.inpe.br -p 2000
mkdir Data_CFSR
exit
sftp -P2000 name.surname@acesso-hpc.cptec.inpe.br
cd Data_CFSR
put SST.tar.gz
put Pressure.tar.gz
put Surface.tar.gz
exit
ssh -Y name.surname@acesso-hpc.cptec.inpe.br -p 2000
cd Data_CFSR
tar -xvzf SST.tar.gz
tar -xvzf Pressure.tar.gz
tar -xvzf Surface.tar.gz
\end{bashcode}
\bigskip

\subsection{\textit{geogrid}}\label{geowps}
\bigskip

 As previously mentioned, \textit{geogrid} generates the grid domain using geographic data. The data can be obtained from WPS 
website\textcolor{bleu_cite}{\textit{}\footnote{\textcolor{bleu_cite}{\href{http://www2.mmm.ucar.edu/wrf/users/download/get\_sources\_wps\_geog.html}{http://www2.mmm.ucar.edu/wrf/users/download/get\_sources\_wps\_geog.html}}}}
 or directly from the repository guide:
\bigskip

\begin{bashcode}
/scratch/name.surname/repositorio/COAWST/WPS/geog
\end{bashcode}
\bigskip

 With the CFSR and \textit{geogrid} data in hand, enter into the WPS directory and open \textit{namelist.wps}. 
The file structure is shown in Figure \textcolor{bleu_cite}{\ref{namelistwps}}:
\bigskip

\begin{bashcode}
nedit namelist.wps
\end{bashcode}
\bigskip

\begin{figure}[H]
    \centering
    \includegraphics[width=0.95\textwidth]{image3411.png}
    \caption{\textit{namelist.wps} example.}
    \label{namelistwps}
\end{figure}
\bigskip

 Change the \textit{geog\_data\_path} path to the directory where geographic data is located (\textit{geog} folder), 
\textit{geog\_data\_res} if you want to use other geographic data and \textit{opt\_geogrid\_tbl\_path}, which is the 
\textit{geogrid} directory.

\bigskip

 The WPS uses a ground zero point defined by the user as the latitude and longitude degrees in \textit{ref\_lat} and \textit{ref\_lon}.
It will generate the grid from that point and through the chosen spatial resolution. Take as example the Figure
\textcolor{bleu_cite}{\ref{namelistwps}}: a simulation will be prepared with 6 km of spatial resolution (\textit{dx} and \textit{dy} = 6000),
using Lambert projection and with 600 grid points. Using as reference latitude -35\degree longitude -45\degree, a grid will be created with 
600 points in the west-east direction (\textit{e\_we}) and 600 points in the south-north direction (\textit{e\_sn}), with each
point having a spacing of 6000 meters (6 km) between them.
\bigskip

 To start \textit{geogrid}, you need a file with a Shell extension (\textit{.sh}), which submits program to be executed in the Kerana cluster.
 It can be found in the \textit{/repository/WPS\_scripts} directory. Move the contents of the folder to the WPS directory with the commands:
\bigskip

\begin{bashcode}[fontsize=\footnotesize]
cd /home/name.surname/repositorio/WPS_scripts
mv qsub_geogrid.sh qsub_ungrib.sh qsub_metgrid.sh /home/name.surname/COAWST/WPS
\end{bashcode}
\bigskip

 The structure of the \textit{qsub\_geogrid.sh} file is shown in Figure \textcolor{bleu_cite}{\ref{qsubgeonedit}}. 
Modify the directory according to your user.
\bigskip

\begin{figure}[H]
    \centering
    \includegraphics[width=0.95\textwidth]{qsubgeonedit.png}
    \caption{\textit{qsub\_geodrid.sh} example.}
    \label{qsubgeonedit}
\end{figure}
\bigskip

 To run \textit{geogrid}, use \textit{qsub} command:
\bigskip

\begin{bashcode}
qsub qsub_geodrid.sh
\end{bashcode}
\bigskip

 Two files will be generated: \textit{log.err} e \textit{log.out}. you can follow how the program is being executed and search for errors. 
The program final message is showned in Figure \textcolor{bleu_cite}{\ref{qsubgeofinal}}, in the file \textit{log.out}.

\bigskip

\begin{figure}[H]
    \centering
    \includegraphics[width=0.55\textwidth]{qsubgeo.png}
    \caption{Example of \textit{geogrid} final message.}
    \label{qsubgeofinal}
\end{figure}
\bigskip

 When \textit{geogrid} is finished, the file \textit{geo\_em\_d01.nc} will be generated.
 You can view the NetCDF file with Ncview:
\bigskip

\begin{bashcode}
module load netcdf
ncview geo_em_d01.nc
\end{bashcode}
\bigskip

 You will need to run \textit{geogrid} until you find a domain that is suitable for your project, 
since WPS does not have a graphical interface. One solution to work around the problem is to use NCAR Command Language version 6.4 
(NCL; \cite{Ncl2017}) to plot the image of the chosen domain. This will be the subject of the \textcolor{bleu_cite}{\ref{nclsec}} subsection.

\bigskip

\subsection{Using the NCL to plot the grid domain}\label{nclsec}
\bigskip

 To install the NCL, move the \textit{NCL} folder inside the repository:
\bigskip

\begin{bashcode}
mv /scratch/name.surname/repositorio/NCL /scratch/name.surname
\end{bashcode}
\bigskip

 Open the \textit{.bashrc} file:
\bigskip

\begin{bashcode}
cd
nedit .bashrc
\end{bashcode}
\bigskip

 Add the following command lines, changing to your username:
\bigskip

\begin{bashcode}
export NCARG_ROOT=/scratch/name.surname/NCL
export PATH=$NCARG_ROOT/bin:$PATH
\end{bashcode}
\bigskip

 Save the file, reload the modules inside \textit{.bashrc} and then execute NCL.
\bigskip

\begin{bashcode}
source .bashrc
ncl
\end{bashcode}
\bigskip

 Check if the NCL has been initialized as in Figure \textcolor{bleu_cite}{\ref{nclinit}}.
\bigskip

\begin{figure}[H]
    \centering
    \includegraphics[width=0.95\textwidth]{nclinit.png}
    \caption{Example of NCL initialization.}
    \label{nclinit}
\end{figure}
\bigskip


 Enter the repository and open the \textit{plotgrids.ncl} file. 
This script generates the domain image from the data in the \textit{namelist.wps} file.
\bigskip

\begin{bashcode}
cd /scratch/name.surname/repositorio
nedit plotgrids.ncl
\end{bashcode}
\bigskip

 Modify the \textit{namelist.wps} directory according to your username:
\bigskip

\begin{bashcode}
filename = "/home/name.surname/COAWST/WPS/namelist.wps"
\end{bashcode}
\bigskip

 Run the code with the command:
\bigskip

\begin{bashcode}
ncl plotgrids.ncl
\end{bashcode}
\bigskip

 The code will search for the grid points and the domain resolution through the file \textit{namelist.wps} and show it on the screen, 
as shown in Figure \textcolor{bleu_cite}{\ref{nclgrids}}. Repeat the process until you find a domain that suits your needs.
\bigskip

\begin{figure}[H]
    \centering
    \includegraphics[width=0.65\textwidth]{wps_domain_new.png}
    \caption{Figure generated by the \textit{plotgrids.ncl} file. In this example, three domains were generated.}
    \label{nclgrids}
\end{figure}
\bigskip

\subsection{\textit{ungrib}}\label{ungribsecao}
\bigskip
 With the domain prepared by \textit{geogrid}, we can proceed to \textit{ungrib}, which extracts the data from the
 \textit {Grib} files. We will start creating the SST data. Open \textit{namelist.wps}
and  change the \textit{prefix} to \textit{SST}, in the \textit{\&ungrib} section, as shown in Figure \textcolor{bleu_cite}{\ref{ungribprefix}}.
\bigskip

\begin{figure}[H]
    \centering
    \includegraphics[width=0.45\textwidth]{ungribprefix.png}
    \caption{ \textit{\&ungrib} section in the \textit{namelist.wps} file.}
    \label{ungribprefix}
\end{figure}
\bigskip

 Import the SST Variable Table using the symbolic link method. Type in the terminal:
\bigskip

\begin{bashcode}
ln -sf ungrib/Variable_Tables/Vtable.SST Vtable
\end{bashcode}
\bigskip

\begin{tcolorbox}[enhanced,
  grow to left by=0cm,%   equivalent to negative mdframed 'leftmargin'
  grow to right by=0cm,%  equivalent to negative mdframed 'rightmargin'
  enlarge top by=0cm,%     equivalent to mdframed 'skipabove'
  enlarge bottom by=0cm,%  equivalent to mdframed 'skipbelow'
  tcbox raise base,
  boxrule=1.0pt,
  left=18mm,
  colframe=red!50!black,coltext=red!25!black,colback=red!10!white,
  overlay={\begin{tcbclipinterior}\fill[red!75!blue!50!white] (frame.south west)
    rectangle node[text=white,font=\sffamily\bfseries\footnotesize,rotate=0] {WARNING} ([xshift=18mm]frame.north west);\end{tcbclipinterior}}]
The \textit{Vtable} is used to read the files in \textit{Grib} format. It is mandatory to use the corresponding \textit{Vtable} to the chosen 
database. You can consult them at \textit{WPS/ungrib/Variable\_Tables folder}.
\end{tcolorbox}
\bigskip

 Create the symbolic links with the SST data through the \textit{link\_grib.csh} file:
\bigskip

\begin{bashcode}
./link_grib.csh /scratch/name.surname/Dados_CFSR/SST/*
\end{bashcode}
\bigskip

 Open the \textit{qsub\_ungrib.sh} file and modify the path according to your user and type:
\bigskip

\begin{bashcode}
qsub qsub_ungrib.sh
\end{bashcode}
\bigskip

\begin{tcolorbox}[enhanced,
  grow to left by=0cm,%   equivalent to negative mdframed 'leftmargin'
  grow to right by=0cm,%  equivalent to negative mdframed 'rightmargin'
  enlarge top by=0cm,%     equivalent to mdframed 'skipabove'
  enlarge bottom by=0cm,%  equivalent to mdframed 'skipbelow'
  tcbox raise base,
  boxrule=1.0pt,
  left=18mm,
  colframe=yellow!50!black,coltext=yellow!25!black,colback=yellow!10!white,
  overlay={\begin{tcbclipinterior}\fill[yellow!75!blue!50!white] (frame.south west)
    rectangle node[text=white,font=\sffamily\bfseries\footnotesize,rotate=0] {ATTENTION} ([xshift=18mm]frame.north west);\end{tcbclipinterior}}]
    The \textit{qsub\_ungrib.sh} file is located in the \textit{/repository/WPS\_scripts} folder.
\end{tcolorbox}
\bigskip

 Several files will be created with the initial \textit {SST:} followed by the date chosen for simulation. 
To check if everything went well, look for the message in Figure \textcolor{bleu_cite}{\ref{ungribsucess}} at the end of the file
\textit{log.out}.
\bigskip

\begin{figure}[H]
    \centering
    \includegraphics[width=0.45\textwidth]{qsubung.png}
    \caption{Final message in the \textit{log.out} file when running \textit{ungrib}.}
    \label{ungribsucess}
\end{figure}
\bigskip

 We will proceed to the CFSR surface data. Change the \textit{prefix} in \textit{namelist.wps} (Figure \textcolor{bleu_cite}{\ref{ungribprefix}}). 
Modify \textit{SST} by \textit{SFC} and save the modification.

\bigskip

 Import the CFSR \textit{Variable Table} with the symbolic link creation command:
\bigskip

\begin{bashcode}
ln -sf ungrib/Variable_Tables/Vtable.CFSR Vtable
\end{bashcode}
\bigskip

 Create the symbolic links to the CFSR surface data with the command below and run \textit{ungrib} again:
\bigskip

\begin{bashcode}
./link_grib.csh /scratch/name.surname/Dados_CFSR/Surface/*
qsub qsub_ungrib.sh
\end{bashcode}
\bigskip

 Look for the final message as shown in Figure \textcolor{bleu_cite}{\ref{ungribsucess}}.
\bigskip

 Change the \textit{prefix} of \textit{namelist.wps} (Figure \textcolor{bleu_cite}{\ref{ungribprefix}}), replace \textit {SFC} with \textit{PRES} and save the file.
\bigskip

 Create the symbolic links to the pressure data and run \textit{ungrib}:
\bigskip

\begin{bashcode}
./link_grib.csh /scratch/name.surname/Dados_CFSR/Pressure/*
qsub qsub_ungrib.sh
\end{bashcode}
\bigskip

 Finally, look for the final message as identified in Figure \textcolor{bleu_cite}{\ref{ungribsucess}}. 
At the end of these steps, there will be several files with the initials \textit {SST}, \textit {PRES}, \textit {SFC} followed by the
dates of the chosen period.
\bigskip


\subsection{\textit{metgrid}}\label{metgridsecao}
\bigskip
 To interpolate the data generated by \textit{ungrib.exe}, open \textit{namelist.wps} and change on the \textit{fg\_name} tab the 
names used in \textit {ungrib}. In the previous case, \textit {PRES}, \textit{SFC} and \textit{SST} were used, as in the example in 
Figure \textcolor{bleu_cite}{\ref{fgname}}:
\bigskip

\begin{figure}[H]
    \centering
    \includegraphics[width=0.5\textwidth]{fgname.png}
    \caption{\textit{\&metgrid} section in the \textit{namelist.wps} file.}
    \label{fgname}
\end{figure}
\bigskip

 Open and change the \textit{qsub\_metgrid.sh} file path and run:
\bigskip

\begin{bashcode}
qsub qsub_metgrid.sh
\end{bashcode}
\bigskip

 Move the files generated by \textit{metgrid} (\textit {met\_em.d01*}) to the real WRF case directory:
\bigskip

\begin{bashcode}
 mv met_em.d01* /home/name.surname/WRF/test/em_real
\end{bashcode}
\bigskip


\begin{tcolorbox}[enhanced,
    grow to left by=0cm,%   equivalent to negative mdframed 'leftmargin'
    grow to right by=0cm,%  equivalent to negative mdframed 'rightmargin'
    enlarge top by=0cm,%     equivalent to mdframed 'skipabove'
    enlarge bottom by=0cm,%  equivalent to mdframed 'skipbelow'
    tcbox raise base,
    boxrule=1.0pt,
    left=18mm,
    colframe=green!50!black,coltext=green!25!black,colback=green!10!white,
    overlay={\begin{tcbclipinterior}\fill[green!75!blue!50!white] (frame.south west)
      rectangle node[text=white,font=\sffamily\bfseries\footnotesize,rotate=0] {NOTE} ([xshift=18mm]frame.north west);\end{tcbclipinterior}}]
If you are an experienced user, you can create symbolic links pointing to the \textit{met} files instead of moving them.
  \end{tcolorbox}
  \bigskip

\subsection{\textit{real}}\label{realsecao}
\bigskip

 We will use the \textit{real.exe} program to generate the files that will be used in WRF. Open the \textit{namelist.input} file, 
which is in the WRF folder (\textit{/home/name.surname/WRF/test/em\_real}), and modify it according to your chosed domain and simulation time in 
\textit{namelist.wps}.
\bigskip

 For the WRF to simulate the SST, it is necessary to include the following variables at the end of the section 
\textit{\&time\_control} (Figure \textcolor{bleu_cite}{\ref{timecontrolnamelist}}):
\bigskip

\begin{bashcode}
io_form_auxinput4  = 2,
auxinput4_inname   = "wrflowinp_d<domain>",
auxinput4_interval = 360,
\end{bashcode}
\bigskip

 The \textit{io\_form\_auxinput4} refers to the final format of the \textit{wrflowinp} file that will be generated by 
the \textit{real}. The \textit{auxinput4\_inname} is the name of the boundary condition file for SST and \textit{auxinput4\_interval} 
is the time interval, in minutes, of the boundary file.
\bigskip

\begin{figure}[H]
    \centering
    \includegraphics[width=0.70\textwidth]{namelisttimecontrol.png}
    \caption{\textit{\&time\_control} section in the \textit{namelist.input} file.}
    \label{timecontrolnamelist}
\end{figure}
\bigskip

 Also, in the \textit{\&physics} section (Figure \textcolor{bleu_cite}{\ref{physicsnamelist}}) in the \textit{namelist.input} 
the option to update the SST:
\bigskip

\begin{bashcode}
sst_update = 1,
\end{bashcode}
\bigskip

\begin{figure}[H]
    \centering
    \includegraphics[width=0.70\textwidth]{physics.png}
    \caption{\textit{\&physics} section in the \textit{namelist.input} file.}
    \label{physicsnamelist}
\end{figure}
\bigskip

 After completing the modifications, you must use the file \textit{qsub\_real.sh} to submit the job.
The file is located in the \textit{/repository/WRF\_scripts} directory. Move the file to the \textit{em\_real} folder, change the directory 
according to your user and execute:
\bigskip

\begin{bashcode}
cd /home/name.surname/repositorio/WRF_scripts
mv qsub_real.sh /home/name.surname/WRF/test/em_real
qsub qsub_real.sh
\end{bashcode}
\bigskip

\begin{tcolorbox}[enhanced,
  grow to left by=0cm,%   equivalent to negative mdframed 'leftmargin'
  grow to right by=0cm,%  equivalent to negative mdframed 'rightmargin'
  enlarge top by=0cm,%     equivalent to mdframed 'skipabove'
  enlarge bottom by=0cm,%  equivalent to mdframed 'skipbelow'
  tcbox raise base,
  boxrule=1.0pt,
  left=18mm,
  colframe=yellow!50!black,coltext=yellow!25!black,colback=yellow!10!white,
  overlay={\begin{tcbclipinterior}\fill[yellow!75!blue!50!white] (frame.south west)
    rectangle node[text=white,font=\sffamily\bfseries\footnotesize,rotate=0] {ATTENTION} ([xshift=18mm]frame.north west);\end{tcbclipinterior}}]
The \textit{qsub\_real.sh} file is located in the \textit{/repository/WRF\_scripts} folder.
\end{tcolorbox}
\bigskip

 To check if \textit{real} succeeded in generating the forcing files, look at the end of the \textit{log.out} file, as shown in Figure \textcolor{bleu_cite}{\ref{realfinish}}.
\bigskip

\begin{figure}[H]
    \centering
    \includegraphics[width=0.70\textwidth]{realsucess.png}
    \caption{Successful completion of \textit{real}.}
    \label{realfinish}
\end{figure}
\bigskip

 Upon completing \textit{real}, three files will be generated: \textit{wrfbdy\_d01}, \textit{wrfinput\_d01} and 
\textit{wrflowinp\_d01}. Now, the WRF conditions are ready to be copied to your project
folder in COAWST:
\bigskip

\begin{bashcode}[fontsize=\scriptsize]
cp wrfbdy_d01 wrfinput_d01 wrflowinp_d01 /home/name.surname/COAWST/Projects/project_name
\end{bashcode}
\bigskip

\section{Using the WRF}\label{wrfsecao2}
\bigskip

 It is possible to start the simulation of your project with the WRF.
\bigskip

 To run the simulation on the cluster, it is necessary to use the file \textit{qsub\_wrf.sh}. It is located in the 
\textit{/repository/WRF\_scripts} folder. Move the file to the WRF \textit{em\_real} folder, modify the directory according
to your username and save:
\bigskip

\begin{bashcode}
cd /home/name.surname/repositorio/WRF_scripts
mv qsub_wrf.sh /home/name.surname/WRF/test/em_real
nedit qsub_wrf.sh
\end{bashcode}
\bigskip

 To start the simulation, type:
\bigskip

\begin{bashcode}
qsub qsub_wrf.sh
\end{bashcode}
\bigskip

\begin{tcolorbox}[enhanced,
  grow to left by=0cm,%   equivalent to negative mdframed 'leftmargin'
  grow to right by=0cm,%  equivalent to negative mdframed 'rightmargin'
  enlarge top by=0cm,%     equivalent to mdframed 'skipabove'
  enlarge bottom by=0cm,%  equivalent to mdframed 'skipbelow'
  tcbox raise base,
  boxrule=1.0pt,
  left=18mm,
  colframe=yellow!50!black,coltext=yellow!25!black,colback=yellow!10!white,
  overlay={\begin{tcbclipinterior}\fill[yellow!75!blue!50!white] (frame.south west)
    rectangle node[text=white,font=\sffamily\bfseries\footnotesize,rotate=0] {ATTENTION} ([xshift=18mm]frame.north west);\end{tcbclipinterior}}]
The \textit{qsub\_wrf.sh} file is located in the \textit{/repository/WRF\_scripts} folder.
\end{tcolorbox}
\bigskip

 Follow the job process using \textit{log.out} and \textit {log.err} files. If you want to have the information 
updated by the terminal, type:
\bigskip

\begin{bashcode}
tail -f log.out
\end{bashcode}
\bigskip
